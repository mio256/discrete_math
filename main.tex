\documentclass[a4paper,11pt]{jsarticle}


% 数式
\usepackage{amsmath,amsfonts}
\usepackage{bm}
% 画像
\usepackage[dvipdfmx]{graphicx}


\begin{document}

\title{Discrete Mathematics II}
\author{mio256}
\date{\today}
\maketitle


\section{ガイダンス 12/9}

\subsection{関係と写像}

集合
$\{a,b,c\}$

順序対
$(a,b,c)$

集合Aと集合Bの直積
$A\times B=\{(a,b)|a\in A,b\in B\}$

$B=\{c,d,e\}$
ならば、
$B\times B=\{(c,c),(c,d),(c,e),
             (d,c),(d,d),(d,e),
             (e,c),(e,d),(e,e)\}$

写像は直積の部分集合

\subsection{最小全域木}

全体が繋がりつつ、コストを抑えたネットワーク

\subsection{最短経路問題}

二点間の最短距離(最短コスト)を計算

\subsection{ネットワークフロー}

始点から終点まで輸送できる最大量を求める

\subsection{マッチング}

端点を共有しない辺を選ぶ

\subsection{彩色問題}

隣り合う2つの点や辺を異なる色に塗る

\section{演習問題}

\subsection{復習問題 1}

$A=\{a,b,c,d\}, B=\{c,d,e\}$

$A\cup B=\{a,b,c,d,e\}$

$A\cap B=\{c,d\}$

$A-B=\{a,b\}$

$B\times B=\{(c,c),(c,d),(c,e),
             (d,c),(d,d),(d,e),
             (e,c),(e,d),(e,e)\}$

\subsection{}

\end{document}